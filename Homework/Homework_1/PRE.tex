\documentclass[letterpaper]{article} 
\usepackage[utf8]{inputenc}
\linespread{0.85}
\usepackage[T1]{fontenc}
\usepackage{amsmath}
\usepackage{amsfonts}
\usepackage{amssymb}
\usepackage{array}
\usepackage{booktabs}
\usepackage{hyperref}
\usepackage[version=4]{mhchem}
\usepackage{stmaryrd}
\usepackage[dvipsnames]{xcolor}
\colorlet{LightRubineRed}{RubineRed!70}
\colorlet{Mycolor1}{green!10!orange}
\definecolor{Mycolor2}{HTML}{00F9DE}
\usepackage{graphicx}
\usepackage{amsmath}
\usepackage{graphicx}
\usepackage{capt-of}
\usepackage{lipsum}
\usepackage{fancyvrb}
\usepackage{tabularx}
\usepackage{listings}
\usepackage[export]{adjustbox}
\graphicspath{ {./images/} }
\usepackage[utf8]{inputenc}
\usepackage[english]{babel}
\usepackage{float}
\usepackage{lipsum}
\usepackage{graphicx}
\usepackage{float}
\usepackage[margin=0.7in]{geometry}
\usepackage{amsmath}
\usepackage{graphicx}
\usepackage{capt-of}
\usepackage{tcolorbox}
\usepackage{lipsum}
\usepackage{graphicx}
\usepackage{float}
\usepackage{listings}
\usepackage{hyperref} 
\usepackage{xcolor} % For custom colors
\lstset{
	language=Python,                % Choose the language (e.g., Python, C, R)
	basicstyle=\ttfamily\small, % Font size and type
	keywordstyle=\color{blue},  % Keywords color
	commentstyle=\color{gray},  % Comments color
	stringstyle=\color{red},    % String color
	numbers=left,               % Line numbers
	numberstyle=\tiny\color{gray}, % Line number style
	stepnumber=1,               % Numbering step
	breaklines=true,            % Auto line break
	backgroundcolor=\color{black!5}, % Light gray background
	frame=single,               % Frame around the code
}
\usepackage{float}
\usepackage[]{amsthm} %lets us use \begin{proof}
	\usepackage[]{amssymb} %gives us the character \varnothing
	
	\title{Homework 1, MATH 4062}
	\author{Zongyi Liu}
	\date{Wed, Jan 28, 2026}
	\begin{document}
		\maketitle
		
		\section{Question 1}
		\textbf{Properties of image}
		
		Let $f: X \rightarrow Y$ and be a function and let $A_{1}, A_{2} \subseteq X$ be sets. Show that
		
		
		\begin{equation*}
			f\left(A_{1} \cap A_{2}\right) \subseteq f\left(A_{1}\right) \cap f\left(A_{2}\right) . \tag{1}
		\end{equation*}
		
		
		Moreover, give an example that in general there is no equality in (1). What happens if we additionally assume that $f$ is injective?
		
		
		\textbf{Answer}
		
		\clearpage
		
		\section{Question 2}
		
		\textbf{Properties of pre-image}
		
		Let $f: X \rightarrow Y$ be a function and let $B_{1}, B_{2} \subseteq Y$ be sets. Show that
		
		\begin{enumerate}
			\item $B_{1} \subseteq B_{2}$ implies that $f^{-1}\left(B_{1}\right) \subseteq f^{-1}\left(B_{2}\right)$.
			\item $f\left(B_{1} \cup B_{2}\right)=f\left(B_{1}\right) \cup f\left(B_{2}\right)$.
			\item $f\left(B_{1} \cap B_{2}\right)=f\left(B_{1}\right) \cap f\left(B_{2}\right)$.
		\end{enumerate}
		
		\textbf{Answer}
		
		\clearpage
		
		\section{Question 3}
		\textbf{Images and pre-images}
		
		Let $f: X \rightarrow Y$ and $A \subseteq X, B \subseteq Y$. Show that
		
		\begin{enumerate}
			\item $A \subseteq f^{-1}(f(A))$.
			\item $f\left(f^{-1}(B)\right) \subseteq B$.
		\end{enumerate}
		
		Give an example to show that in general there is no equality.
		
		\textbf{Answer}
		
		\clearpage
		\section{Question 4}
		\textbf{Complement of sets}
		
		Let $A, B \subseteq X$ be sets. Show that
		
		\begin{enumerate}
			\item $(A \cup B)^{C}=A^{C} \cap B^{C}$.
			\item $(A \cap B)^{C}=A^{C} \cup B^{C}$.
			\item If $A \subseteq B$, then $B^{C} \subseteq A^{C}$.
		\end{enumerate}
		
		\textbf{Answer}
		
		\clearpage
		\section{Question 5}
		\textbf{Injective and surjective functions}
		
		Let $X=\{1,2,3\}$. Is there a function $f: X \rightarrow X$ which is injective but not surjective? Explain your answer.
		
		
		\textbf{Answer}
		
		\vspace*{0.1\textheight}
		
		\section{Question 6}
		
		\textbf{Properties of real numbers}
		
		Using the axioms (A1)-(A4), (M1)-(M4), (D), (O1)-(O4), (OC1)-(OC2), (C) of the real numbers, show the following statements:
		
		\begin{enumerate}
			\item The neutral element of addition is unique.
			\item The additive inverse is unique.
			\item The neutral element of multiplication is unique.
			\item The multiplicative inverse is unique.
			\item We have $-(-x)=x$.
			\item Let $x \in \mathbb{R}$. Show that $x \geq 0$ if and only if $(-x) \leq 0$.
			\item Let $x, y, u, v \in \mathbb{R}$ with $x \leq y$ and $u \leq v$. Show that $x+u \leq y+v$.
			\item Let $x, y, z \in \mathbb{R}$ with $z>0$. Show that $x z \leq y z$ implies that $x \leq y$.
		\end{enumerate}
		
		
		\textbf{Answer}
		
		\clearpage
		
		
		\section{Question 7}
		
		\textbf{Existence of infimum}
		
		Let $X \subseteq \mathbb{R}$ be a non-empty set which is bounded from below. Show that the infimum of $X$ exists.
		
		\textbf{Answer}
		
		\vspace*{0.2\textheight}
		
		
		\section{Question 8}
		
		\textbf{Equivalence of least upper bound property and completeness}
		
		Assume that ($\mathbb{R},+, \cdot, \leq$) satisfy the axioms (A1)-(A4), (M1)-(M4), (D), (O1)-(O4), (OC1)-(OC2). Moreover, suppose that the supremum of non-empty sets which are bounded from above exists. Show that $(\mathbb{R},+\cdot, \leq)$ also satisfies the completeness property (C).
		
		\textbf{Answer}
		
	\end{document}
